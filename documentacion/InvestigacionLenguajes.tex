% !TEX TS-program = pdflatex
% !TEX encoding = UTF-8 Unicode

% This is a simple template for a LaTeX document using the "article" class.
% See "book", "report", "letter" for other types of document.

\documentclass[12pt]{book} % use larger type; default would be 10pt

\usepackage[utf8]{inputenc} % set input encoding (not needed with XeLaTeX)

%%% Examples of Article customizations
% These packages are optional, depending whether you want the features they provide.
% See the LaTeX Companion or other references for full information.

%%% PAGE DIMENSIONS
\usepackage[right=2cm,left=3cm,top=2cm,bottom=2cm,headsep=0cm,footskip=0.5cm]{geometry}
\usepackage{geometry} % to change the page dimensions
\geometry{a4paper} % or letterpaper (US) or a5paper or....
% \geometry{margin=2in} % for example, change the margins to 2 inches all round
% \geometry{landscape} % set up the page for landscape
%   read geometry.pdf for detailed page layout information

\usepackage{graphicx} % support the \includegraphics command and options

% \usepackage[parfill]{parskip} % Activate to begin paragraphs with an empty line rather than an indent

%%% PACKAGES
\usepackage{booktabs} % for much better looking tables
\usepackage{array} % for better arrays (eg matrices) in maths
%\usepackage{paralist} % very flexible & customisable lists (eg. enumerate/itemize, etc.)
\usepackage{verbatim} % adds environment for commenting out blocks of text & for better verbatim
\usepackage{subfig} % make it possible to include more than one captioned figure/table in a single float
% These packages are all incorporated in the memoir class to one degree or another...

%%% HEADERS & FOOTERS
\usepackage{fancyhdr} % This should be set AFTER setting up the page geometry
\pagestyle{fancy} % options: empty , plain , fancy
\renewcommand{\headrulewidth}{0pt} % customise the layout...
\lhead{}\chead{}\rhead{}
\lfoot{}\cfoot{\thepage}\rfoot{}

%%% SECTION TITLE APPEARANCE
\usepackage{sectsty}
\allsectionsfont{\sffamily\mdseries\upshape} % (See the fntguide.pdf for font help)
% (This matches ConTeXt defaults)

%%% ToC (table of contents) APPEARANCE
\usepackage[nottoc,notlof,notlot]{tocbibind} % Put the bibliography in the ToC
\usepackage[titles,subfigure]{tocloft} % Alter the style of the Table of Contents
\renewcommand{\cftsecfont}{\rmfamily\mdseries\upshape}
\renewcommand{\cftsecpagefont}{\rmfamily\mdseries\upshape} % No bold!

%%% Definiendo nuevos COLORES
\usepackage{xcolor} %Paquete de Color 
\definecolor{verde}{rgb}{0.25,0.5,0.35}
\definecolor{jpurple}{rgb}{0.5,0,0.35}

%%% Configurando el Layout para mostrar codigos Java
\usepackage{listings}
\lstset{
  language=Java,
  basicstyle=\ttfamily\small,
  keywordstyle=\color{jpurple}\bfseries,
  stringstyle=\color{red},
  commentstyle=\color{verde},
  morecomment=[s][\color{blue}]{/**}{*/},
  extendedchars=true,
  showspaces=false,
  showstringspaces=false,
  numbers=left,
  numberstyle=\tiny,
  breaklines=true,
  backgroundcolor=\color{cyan!10},
  breakautoindent=true,
  captionpos=b,
  xleftmargin=0pt,
  tabsize=3
}

%%% END Article customizations

\usepackage[spanish]{babel}
\usepackage{listings} 
\newcommand{\HRule}{\rule{\linewidth}{0.5mm}}
%%% The "real" document content comes below...


\title{Parsing XML with Haskell}
\author{Adriana Rodr\'iguez S\'anchez}

%\date{} % Activate to display a given date or no date (if empty),
         % otherwise the current date is printed 


\usepackage{eso-pic}
\newcommand\BackgroundEspol{
	\put(-218,338){
	\parbox[b][\paperheight]{\paperwidth}{%
           	\vfill
	           \centering
           	\includegraphics[height=0.07\textheight,width=0.3\textwidth,keepaspectratio]{logoespol.jpg}%
	            \vfill
}}}

\newcommand\BackgroundFiec{
	\put(245,338){
	\parbox[b][\paperheight]{\paperwidth}{%
           	\vfill
	           \centering
           	
	            \vfill
}}}


\begin{document}


\begin{titlepage}
%\hspace*{0.2in}
\AddToShipoutPicture*{\BackgroundEspol}
\AddToShipoutPicture*{\BackgroundFiec}
%\includegraphics[height=0.1\textheight]{logoespol}
%\hspace*{2.0in}
%\includegraphics[width=0.3\textwidth]{logofiec}
%\includegraphics[width=0.1\textwidth]
\begin{center}
\textsc{\Large Escuela Superior Politécnica del Litoral}\\[1.0cm]
\textsc{\Large Facultad de Ingenieria en Electricidad y Computación}\\[1.5cm]

% Titulo
\HRule \\[0.4cm]
{ \LARGE \bfseries Parsing XML with Haskell \\[0.4cm] }
\HRule \\[1.5cm]


% Autores
\begin{minipage}{0.4\textwidth}
\begin{flushleft} \large
\emph{Autor:}\\
Adriana \textsc{Rodríguez}\\
\end{flushleft}
\end{minipage}
\begin{minipage}{0.4\textwidth}
\begin{flushright} \large
\emph{Profesor:} \\
Ing.~Javier \textsc{Tibau}
\end{flushright}
\end{minipage}
\vfill
% Bottom of the page
{\large \today}
\end{center}
\end{titlepage}





%\maketitle
\tableofcontents

\newpage
\mbox{}

\chapter{Introducci\'on}
XML es un lenguaje de marcado que permite almacenar datos, para ello esta conformado de una estructura abstracta para que as\'i se puedan trabajar con datos grandes.

En este caso, el prop\'osito de este documento es dar a conocer como en el lenguaje Haskell se puede trabajar para la administraci\'on de archivos XML y as\'i mostrar datos que sean espec\'ificos para el usuario. 

Para llevar un mejor control en los datos, decidimos trabajar por medio del \textit{parsing} que permite que los datos sean trabajados de una manera mas textual en la observaci\'on de todo el documento en si.

\begin{figure}[h!]
        \begin{center}
        \includegraphics[scale=0.5]{logo.jpeg}
        \end{center}
        \caption{Logo de Haskell.}
\end{figure}~\\[2cm]



\chapter{Alcance}
El alcance del proyecto es poder realizar un procesamiento de la informaci\'on que est\'a contenida en el archivo \textit{wurfl23.xml} y mostrar datos correspondientes a los dispositivos que cumplan con ciertas caracter\'isticas dichas por el usuario.

\begin{figure}[h!]
        \begin{center}
        \includegraphics[scale=0.9]{ejemplo.jpg}
        \end{center}
        \caption{Ejemplo del capability.}
\end{figure}~\\[2cm]

\chapter{Descripci\'on}
El objetivo principal del proyecto es poder comprender como trabaja Haskell, como interpreta los datos y las funcionalidades que este posee. Iniciando por la recursividad con la que este lenguaje trabaja, para ello crearemos ciertas funciones que nos ayuden a manejar los datos correctamente; y as\'i manipularlos manualmente.~\\[0.5cm]

Los archivos XML trabajan por medio del marcado, llevan un control de los grupos de informaci\'on para asi mostrarlos como \textit{elementosl} que proporcionan una lista de estos elementos. El archivo antes mencionado, est\'a conformado por una serie elementos principales, estos son: Device, Group y los capability que son las listas principales que nos ayudar\'an a manipular los datos.~\\[0.5cm]

En el caso de los Device est\'an conformado por: 
\begin{itemize}
  \item id
  \item user agent
  \item fall back
\end{itemize}

A su vez, los Group est\'an conformado por: 
\begin{itemize}
  \item id
\end{itemize}

Y por \'ultimo los capability est\'an conformado por: 
\begin{itemize}
  \item name
  \item value
\end{itemize}


\chapter{Implementaci\'on}
El proyecto est\'a compuesto inicalmente por la estructura principal del xml, Device, Group y Capability con sus correspondientes datos, ayudandonos por el mismo medio se encuentre Grupo que es un dato que esta conformado por los tres elementos principales.~\\[0.5cm]
 Cada elemento, tiene su respectivo create, y a su vez tiene una funcio\'on que permite listarlos de manera indenpendiente. Sin embargo, lo dificultoso de esto fue, la interpretaci\'on de los datos como manejarlos y entender de manera mas abstracta la recursividad.~\\[0.5cm]

\begin{figure}[h!]
        \begin{center}
        \includegraphics[scale=0.7]{ejemplo2.jpg}
        \end{center}
        \caption{Ejemplo de la codificaci\'on.}
\end{figure}~\\[2cm]
\chapter{Observaciones}
La manipulaci\'on del nuevo lenguaje no fue para nada f\'acil de utilizarlo, se tuvo que investigar bastante y proporcionarse de varios ejemplos para poder entenderlo. Sin embargo, Haskell es un lenguaje que nos ayuda a manejar los datos de una manera mas sencilla sin la necesidad de las condiciones de recursividad, sino que uno mismo debe realizarlo independientemente. La herramienta ayuda al programador a tener una idea mas amplia de como estas funciones son implementadas y que existen diferentes maneras de realizar un proyecto en diferentes lenguajes de programaci\'on.
\chapter{Conclusiones}
Haskell de cierta forma nos ayud\'o en: 
\begin{itemize}
  \item Comprender como trabajan los lenguajes funcionales.
  \item Trabajar y entender la recursividad.
  \item Establecer metas claras de cuales son las funciones principales.
  \item Mejor entendimiento del manejo del parsing.
\end{itemize}
\end{document}
